\clearpage{\pagestyle{empty}\cleardoublepage}
\chapter{Il problema}\label{Chap:Prob}

\section{Definizione}

Il problema affrontato � l'accoppiamento a costo minimo con vincoli di risorsa.

Si devono assegnare $n$ persone (o macchine o entit�) a $n$ progetti (o lavori), in modo che ciascuna persona sia assegnata a un progetto diverso. Assegnare la persona $i$ al progetto $j$ comporta un costo $c_{ij}$, ma anche l'uso di una risorsa in quantit� limitata $r_{ij}$.

Il problema pu� essere definito tramite un grafo bipartito $G=(V_{1} \cup V_{2},E)$ dove il sottoinsieme di vertici $V_{1}$ rappresenta l'insieme delle persone, il sottoinsieme di vertici $V_{2}$ rappresenta l'insieme dei progetti, mentre l'insieme dei lati $E$ rappresenta gli accoppiamenti possibili tra persone e progetti.\\
Si vuole minimizzare il costo totale di tutti gli assegnamenti, garantendo che l'uso totale della risorsa non superi una data disponibilit� $B$ (\emph{Budget}).

\section{Formulazione}

\textsc{Dati}:
\begin{itemize}
	\item $|V_{1}|=n$ persone e $|V_{2}|=n$ progetti
	\item $c_{ij}$ costo di attribuzione progetto $j \in V_{2}$ alla persona $i \in V_{1}$
	\item $r_{ij}$ risorsa consumata dalla persona $i \in V_{1}$ per il fare il progetto $j \in V_{2}$
		%3)
	\item L'uso di risorse totale:\hfill
		\begin{equation}
			R = \displaystyle\sum_{i \in V_{1}}\ds\sum_{j \in V_{2}} r_{ij}
		\end{equation}
	%4)
	\item Il budget $B$, viene calcolato come ($\rho$ � un parametro):\hfill
		\begin{equation}\label{eq:budget_cal}
			B = \displaystyle\frac{R}{n} \cdot \rho \qquad \rho \in \left\{ 0.6,0.8,0.1 \right\}
		\end{equation}
\end{itemize}
\textsc{Variabili}:
	\begin{itemize}
		\item Una soluzione � individuata dalle variabili:
	\end{itemize}
	\begin{equation*}
		x_{i,j}=
			\begin{cases}		
				1& \text{progetto $j \in V_{2}$ alla persona $i \in V_{1}$},\\		
				0& \text{altrimenti}.
			\end{cases}
	\end{equation*}
\textsc{Vincoli}:
\begin{itemize}
	%1)
	\item Una persona deve essere associata a un solo progetto:\hfill
		\begin{equation}\label{eq:prog_lav0}
			\displaystyle\sum_{j \in V_{2}}x_{ij}=1 \qquad i \in V_{1}
		\end{equation}
	%2)
	\item Un progetto deve essere associato a una sola persona:\hfill
		\begin{equation}\label{eq:lav_prog0}
			\displaystyle\sum_{i \in V_{1}}x_{ij}=1\qquad j \in V_{2}
		\end{equation}
	%3)
	\item Vincolo di budget\hfill
		\begin{equation}\label{eq:budget0}
			\ds\sum_{i \in V_{1}}\ds\sum_{j \in V_{2}} r_{ij} x_{ij} \leq B
		\end{equation}
\end{itemize}
\textsc{Obiettivo}:
\begin{itemize}
	\item Si vuole minimizzare il costo totale:
		\begin{equation}
			\min\ \ds\sum_{i \in V_{1}}\ds\sum_{j \in V_{2}} c_{ij} x_{ij} 
		\end{equation}
\end{itemize}
